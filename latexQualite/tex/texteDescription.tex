%!TEX encoding = UTF-8 Unicode

\chapter*{Information concernant la remise 1}

\section*{Liste des scénarios avec leur statut}
\begin{enumerate}
	\item (U01) Envoyer les services d'urgence en cas d'intrusion: Terminé
	\item (C01) Armer via le clavier: Terminé
	\item (C02) Désarmer avec un NIP au clavier: Terminé
	\item (U02) Envoyer les services d'urgence en cas de fumée: Terminé
	\item (U07) Obtenir la liste des alertes: Terminé
	\item (U03) Pouvoir signaler manuellement une urgence (bouton panique): Non commmencé
	\item (U04) Envoyer les services médicaux en cas de demande d'assistance médicale: Non commmencé
	\item (C03) Changer son NIP via le clavier: Terminé
	\item (D01) Envoyer un courriel en cas d'alarme: Non commmencé
	\item (U05) Détecter les températures élevées: Non commmencé
	\item (D02) Armer/désarmer à partir de l'app. mobile: Non commmencé
	\item (D03) Voir les caméras sur son téléphone: Non commmencé
	\item (D04) Signaler un oubli de barrer la porte au départ du dernier occupant: Non commmencé
	\item (D05) Débarrer et ouvrir les portes en cas de feu: Non commmencé
	\item (U06) Gérer une intrusion pendant Noël: Non commmencé
	\item (D06) Fermer les lumières en cas d'intrusion: Non commmencé
\end{enumerate}

\section*{Comment démarrer/utiliser votre système}
Il est possible de démarrer le système avec Maven. D'abord, il faut se rendre dans le dossier du \emph{workspace} (là où se trouvent tous les dossiers des projets) et taper \emph{mvn install} en ligne de commande pour installer l'ensemble des projets. Il est à noter que cette commande exécute également les tests unitaires et les tests d'acceptation.\\
S'il est nécessaire d'exécuter les tests à nouveau, il est possible de taper \emph{mvn test}, toujours en ligne de commande, dans le dossier du \emph{workspace} ou encore dans chaque dossier des projets afin d'exécuter les tests unitaires d'un projet en particulier.\\
Pour exécuter les tests d'acceptation, on peut taper \emph{mvn integration-test -pl acceptanceTests}.

Pour démarrer les serveurs, il suffit de taper les commandes suivantes, selon le cas :\\
\emph{mvn exec:java -pl centralServer}\\
\emph{mvn exec:java -pl emergencyServer}

\section*{Description de tous vos protocoles entre les composantes}
Entre le client et le serveur de la compagnie du système d'alarme, nous avons utilisé un protocole REST. L'adresse utilisée présentement pour ce serveur est le \emph{localhost:9001/}. Les usagers qui veulent s'enregistrer aux services qu'offre la compagnie doivent transmettre l'adresse de leur domicile à l'aide d'une requête POST sur le chemin /register/, encodée avec JSON. Cette étape se fait à l'initialisation du système d'alarme. Suite à cette requête, un identifiant leur est retourné. En cas d'intrusion, les clients peuvent donc ensuite faire une requête GET à l'adresse /client/\{userid\}/police (où \{userid\} est l'identifiant qui leur a été préalablement retourné lors de l'inscription). \\

Toujours en cas d'intrusion, le serveur de la compagnie de systèmes d'alarme pourra ensuite contacter l'urgence en effectuer une requête POST au chemin /police/ en envoyant les informations du client. Il est à noter que pour l'instant, l'adresse pour l'urgence est \emph{localhost:9002/}.

\section*{Toutes notes que vous désirez transmettre aux correcteurs}
\begin{itemize}
	\item Comme certains scénarios requièrent parfois d'attendre jusqu'à trente secondes, les tests d'acceptation mettent un certain temps à s'exécuter (environ une minute). Toutefois, les autres tests devraient s'exécuter en quelques secondes.
	\item Dans la classe ca.ulaval.glo4002.centralServer.core.rest.RegiterResource, nous avons décidé de générer le int qui correspond au UserID et d'inscrire le client en deux étapes. Nous avons fait cela afin d'éviter de violer le CQS. Comme nous devons retourner l'ID générer pour le client il nous faut le retourner dans une méthode. En faisant le tout en une seule méthode, cette méthode aurait généré un ID et inscrit le client ce qui ne respecte pas le CQS. 
	\item Dans la classe ca.ulaval.glo4002.centralServer.core.communication.Communicator, nous ne traitons pas l'exception dans la méthode createMessageToSend puisqu'il n'est pas possible que cette exception soit lancée. En effet, celle-ci est lancée lorsque les clés ou les valeurs ne sont pas des strings. Afin de ne pas faire remonter l'exception, nous avons décidé de lancer notre propre exception qui hérite de RuntimeException.
	\item La classe ca.ulaval.glo4002.domain.alarmSystem.AlarmSystem ne contrôle que l'état du système. Elle implémente un DelayTimerDelegate, c'est-à-dire qu'elle est un objet qui reçoit un message d'un timer après que celui-ci ait atteint 0.
	\item La classe ca.ulaval.glo4002.domain.policies.Policy ne définit pas de méthode abstraite parce qu'elle implémente la méthode executeProcedure qui est identique pour la majorité des classes enfants. Cependant, certains objets qui implémentent Policy ont des procédures plus précises et cette méthode doit être redéfinie. Bref, c'est principalement afin d'éviter la duplication de code que ceci a été fait. Aussi, la méthode de la classe mère prend un int qui correspond à la zone en argument, mais ne l'utilise pas. Ce int est utilisé justement dans les procédures un peu plus complète. Nous savons que ce n'est pas une solution parfaite, mais aucune solution ne l'était.
	\item Nos tests d'acceptation vérifient les différentes fonctionnalités implémentées. De ce fait, ils servent aussi de tests de fonctionnalités.
	\item La plupart de nos tests d'acceptation couvrent l'ensemble du projet. De ce fait, ils servent aussi de tests end-to-end.	
\end{itemize}
